% -- Anfang Präambel
\documentclass[german,  % Standardmäßig deutsche Eigenarten, englisch -> english
parskip=full,  % Absätze durch Leerzeile trennen
%bibliography=totoc,  % Literatur im Inhaltsverzeichnis (ist unüblich)
%draft,  % TODO: Entwurfsmodus -> entfernen für endgültige Version
]{scrartcl}
\usepackage[utf8]{inputenc}  % Kodierung der Datei
\usepackage[T1]{fontenc}  % Vollen Umfang der Schriftzeichen
\usepackage[ngerman]{babel}  % Sprache auf Deutsch (neue Rechtschreibung)

% Mathematik und Größen
\usepackage{amsmath}
\usepackage[locale=DE,  % deutsche Eigenarten, englisch -> US
separate-uncertainty,  % Unsicherheiten seperat angeben (mit ±)
]{siunitx}
\usepackage{physics}  % Erstellung von Gleichungen vereinfachen
\usepackage{yfonts}  % Frakturschrift für Real- und Imaginärteil komplexer Größen

\usepackage{graphicx}  % Bilder einbinden \includegraphics{Pfad/zur/Datei(ohne Dateiendung)}

% Gestaltung
%\usepackage{microtype}  % Mikrotypographie (kann man am Ende verwenden)
\usepackage{booktabs}  % schönere Tabellen
\usepackage[toc]{multitoc}  % mehrspaltiges Inhaltsverzeichnis
\usepackage{csquotes}  % Anführungszeichen mit \enquote
\usepackage{caption}  % Anpassung der Bildunterschriften, Tabellenüberschriften
\usepackage{subcaption}  % Unterabbildungen, Untertabellen, …
\usepackage{enumitem}  % Listen anpassen
\setlist{itemsep=-10pt}  % Abstände zwischen Listenpunkten verringern

% Manipulation des Seitenstils
\usepackage{scrpage2}
% Kopf-/Fußzeilen setzen
\pagestyle{scrheadings}  % Stil für die Seite setzen
\clearscrheadings  % Stil zurücksetzen, um ihn neu zu definieren
\automark{section}  % Abschnittsnamen als Seitenbeschriftung verwenden
\ofoot{\pagemark}  % Seitenzahl außen in Fußzeile
\ihead{\headmark}  % Seitenbeschriftung mittig in Kopfzeile
\setheadsepline{.5pt}  % Kopzeile durch Linie abtrennen

\usepackage[hidelinks]{hyperref}  % Links und weitere PDF-Features

% TODO: Titel und Autor, … festlegen
\newcommand*{\titel}{Quantenanalogie}
\newcommand*{\autor}{Tom Drechsler, Konstantin Schmid}
\newcommand*{\abk}{QA}
\newcommand*{\betreuer}{<Betreuer>}
\newcommand*{\messung}{22.11.2019}
\newcommand*{\ort}{<Ort>}

\hypersetup{pdfauthor={\autor}, pdftitle={\titel}}  % PDF-Metadaten setzen

% automatischen Titel konfigurieren
\titlehead{Fortgeschrittenen-Praktikum \abk \hfill TU Dresden}
\subject{Versuchsprotokoll}
\title{\titel}
\author{\autor}
\date{\begin{tabular}{ll}
Protokoll: & \today\\
Messung: & \messung\\
Ort: & \ort\\
Betreuer: & \betreuer\end{tabular}}

% -- Ende Präambel

\begin{document}
\begin{titlepage}
\maketitle  % Titel setzen
\tableofcontents  % Inhaltsverzeichnis
\end{titlepage}

% Text Anfang
\section{Versuchsziel und Überblick}
Ziel des Versuchs ist es Quantenphysik einfach zu erklären und anschaulich zu interpretieren. Grundlage dafür bietet die Überlegung, dass Elektronen Materiewellen sind und bei diesen somit auch die Phänomene Reflexion, Beugung, Brechung, Überlagerung, Interferenz und stehende Wellen auftreten. Somit sollte es auch möglich sein dies mit Wellen, deren Wellenlängen in einem vergleichsweise makroskopischen Bereich liegen (für z.B Schall einige cm), zu rekonstruieren.
\newline Mit Hilfe eines Lautsprechers sollen Schallwellen erzeugt werden, die in einen Resonator eingesperrt werden. Dieser hat je nach zu betrachtender Analogie eine andere Form, die im jeweiligen Versuchsunterpunkt diskutiert werden wird. Danach wird die Ausbildung stehender Wellen untersucht, da diese unter gleichen Randbedingungen die jeweilige quantenmechanische Lösung gut modellieren. Die gebundenen Lösungen der Schrödinger-Gleichung sind nämlich stehende Wellen.

\section{Theoretische Grundlagen}

\subsection{Quantenmechanisches Teilchen in einer Box}
In diesem Teilversuch wird ein Röhrenresonator verwendet. Dies ist ein geschlossener Hohlzylinder der Länge $L$ mit einem Lautsprecher an einem Ende. Bei einige Frequenzen $f$ stellen sich stehende Wellen ein. Dies kann durch folgende Gleichung, die sogenannte Resonanzbedingung, beschrieben werden:
\begin{align}
2L = n \, \frac{c}{f} = n\lambda
\end{align}
Hierbei ist $c$ die Schallgeschwindigkeit, $\lambda$ die Wellenlänge und $n  \in N_{>0}$. Dies kann Analog zu den Lösungen für ein quantenmechanisches Teilchen in einem unendlich hohen Potentialtopf betrachtet werden.
\newline
\newline Nun gilt es sich die Differentialgleichungen und deren Lösungen beider Situationen anzuschauen, um diese zu vergleichen und sich so die Qualität und den Gültigkeitsbereich der Analogie klarzumachen.
\newline
\newline Die Wellengleichung für den Luftdruck resultiert aus der linearisierten Eulergleichung und der Kontinuitätsgleichung und lautet:
\begin{align}
\frac{\partial^2p}{\partial t^2}=\frac{1}{\rho \kappa} \Delta p
\end{align}
$\rho$ ist die Massendichte der Luft, $t$ die Zeit, $\kappa$ die Kompressibilität und $p$ der Luftdruck. Als Ansatz zur Lösung dieser partiellen Differentialgleichung verwendet man:
\begin{align}
p(x)=p_0 \, \cos(kx-\omega t +\alpha)
\end{align}
Es wurde dabei bereits angenommen, dass ein quasi-eindimensionales Problem betrachtet wird. $p_0$ ist dabei die Amplitude. 
Nun muss man noch berücksichtigen, dass eine Überlagerung von nach rechts und nach links laufender Welle stattfindet. Die Funktion lautet daher:
\begin{align}
p(x)=\frac{1}{2}\,p_0 \, \cos(kx-\omega t -\alpha)
\end{align}
Umgeschrieben resultiert dann:
\begin{align}
p(x)=p_0 \, \cos(kx+\alpha)\, \cos(\omega t)
\end{align}
Durch Betrachtung der Randbedingungen $\frac{dp}{dx}\left(0\right)=0$ und $\frac{dp}{dx}\left(L\right)=0$ ergibt sich $\alpha=0$ und $k =\frac{n \pi}{L}$.
\newline
\newline In der Quantenmechanik betrachten wir die zeitunabhängige Schrödinger-Gleichung:
\begin{align}
\label{sgl}E\Psi(\vec{r}) = -\frac{\hbar^2}{2m} \Delta \Psi(\vec{r}) + V(\vec{r})\Psi(\vec{r},t)
\end{align}
Für den eindimensionalen unendlich hohen Potentialtopf ist das Potential $V(x)$ definiert durch:
\begin{align}
V(x)=
  \begin{cases}
    0 & \text{für }|x|<\frac{L}{2}\\
	\infty & \text{für }|x| \geq \frac{L}{2}
  \end{cases}
\end{align}
Damit reduziert sich (\ref{sgl}) innerhalb des Potentialtopfes zu:
\begin{align}
E\Psi(x) = -\frac{\hbar^2}{2m} \Delta \Psi(x)
\end{align}
Die Lösungen davon haben die Form:
\begin{align}
\Psi(x)=\sqrt{\frac{2}{L}}\,\sin(kx+\alpha)
\end{align}
Durch Multiplikation mit einem Phasenfaktor erhält man daraus die zeitabhängige Lösung:
\begin{align}
\Psi(x,t)=\sqrt{\frac{2}{L}} \, \sin(kx+\alpha) \, e^{i \omega t}
\end{align}
%hier fehlt noch der vergleich

\subsection{Analogon zum Wasserstoffatom}
Für den zweiten Teilversuch wird ein Kugelresonator benötigt. Dieser besteht aus zwei Halbkugeln. In der oberen ist ein Mikrophon integriert und in der unteren ein Lautsprecher. Die beiden Hemisphären kann man gegeneinander verdrehen und dabei die Amplituden in Abhängigkeit vom Winkel messen.
\newline
\newline In der Quantenmechanik können wir als Ansatz für das Potential in der Schrödinger-Gleichung nun das Coulomb-Potential $-\frac{e^2}{r}$ verwenden, das es sich um ein Ein-Elektronen-System handelt. Zur Lösung muss man Kugelkoordinaten einführen und den Separationsansatz $\psi(r, \theta, \varphi) = Y^{m}_{l}(\theta, \varphi) \, R_{l}(r)$ betrachten. Daraus ergeben sich zwei Gleichungen:
\begin{align}
\label{kugelfl}-\bigg \lbrack \frac{1}{\sin(\theta)} \frac{\partial}{\partial \theta} \bigg(\sin(\theta) \frac{\partial}{\partial \theta} \bigg) + \frac{1}{\sin^2(\theta)} \frac{\partial^2}{\partial \varphi^2} \bigg \rbrack Y^{m}_{l}(\theta, \varphi) &= l(l+1) Y^{m}_{l}(\theta, \varphi) \\
-\frac{\hbar^2}{2mr} \frac{\partial^2}{\partial r^2} r R(r) - \frac{l(l+1)\hbar^2}{2mr^2} R(r)-\frac{e^2}{r} &= ER(r)
\end{align}
Die Lösungen der ersten, winkelabhängigen Gleichung sind die Kugelflächenfunktionen.
\newline
\newline Als Gleichung für den Druck  müssen wir die Helmholtz-Gleichung betrachten:
\begin{align}
-\frac{\omega^2}{c^2} \, p(\vec{r}) = \Delta p(\vec{r})
\end{align}
Dies wird nach Umformung in Kugelkoordinaten und mit dem Separationsansatz $p(r, \theta, \varphi) = Y^{m}_{l} \, f(r)$ zu:
\begin{align}
\label{kugelfl}-\bigg \lbrack \frac{1}{\sin(\theta)} \frac{\partial}{\partial \theta} \bigg(\sin(\theta) \frac{\partial}{\partial \theta} \bigg) + \frac{1}{\sin^2(\theta)} \frac{\partial^2}{\partial \varphi^2} \bigg \rbrack Y^{m}_{l}(\theta, \varphi) &= l(l+1) Y^{m}_{l}(\theta, \varphi) \\
-\frac{\partial^2f}{\partial r^2}-\frac{2}{r}\frac{\partial f}{\partial r}+\frac{l(l+1)}{r^2}\, f(r) &= \frac{\omega^2}{c^2}\, f(r)
\end{align}
Die erste Gleichung hat die gleiche Form wie Gleichung (\ref{kugelfl}) und haben somit dieselben Lösungen, die Kugelflächenfunktionen. Die Radialgleichung unterscheiden sich allerdings. Die Kugelflächenfunktionen können durch die assoziierten Legendre Polynome $P^{m}_{l}$ dargestellt werden:
\begin{align}
 Y^{m}_{l} 	\propto P^{m}_{l}(\cos(\theta))\, e^{im\varphi}
\end{align}
In diesem Versuch reicht es aufgrund der Zylindersymmetrie der vom Lautsprecher ausgesendeten Wellen aus den Fall $m = 0$ zu betrachten. $m$ ist hierbei die magnetische Quantenzahl.

\section{Aufbau und Versuchsprinzip}







    % Bibliographie/Literaturverzeichnis
    \begin{thebibliography}{9}
    \bibitem{wiki:Intensität}
    Wikipedia,
    \emph{Intensität (Physik)},
    \url{https://de.wikipedia.org/wiki/Intensit%C3%A4t_(Physik)},
    25.\,Okt.~2019.
    \end{thebibliography}
% Ende Dokument

\end{document}